\documentclass{article}
\usepackage[utf8]{inputenc}
\usepackage{amsmath,amssymb} % for curved letters (\mathcal)
\usepackage{stmaryrd}        % for double brackets
\usepackage{tabularx}        % for aligning the table


\title{Software Verification - Exercises}
\author{Alberto Schiabel}
\date{October 2019}

\begin{document}

\maketitle

% n \in \textit{Num}
% x \in \textit{Var}
% a \in \textit{Aexp}
\begin{table}[ht!]
 \centering
     \setlength{\tabcolsep}{1em} % for the horizontal padding
    \begin{tabular}{|p{\textwidth}|}
    \hline
    $\begin{aligned}[t] % placement: default is "center", options are "top" and "bottom"
    \mathcal{A}\llbracket n \rrbracket s = \mathcal{N} \llbracket n \rrbracket \\
    \mathcal{A}\llbracket x \rrbracket s = s x \\
    \mathcal{A}\llbracket a_1 + a_2 \rrbracket s = \mathcal{A} \llbracket a_1 \rrbracket s + \mathcal{A} \llbracket a_2 \rrbracket s \\
    \mathcal{A}\llbracket a_1 * a_2 \rrbracket s = \mathcal{A} \llbracket a_1 \rrbracket s * \mathcal{A} \llbracket a_2 \rrbracket s \\
    \mathcal{A}\llbracket a_1 - a_2 \rrbracket s = \mathcal{A} \llbracket a_1 \rrbracket s - \mathcal{A} \llbracket a_2 \rrbracket s \\
    \end{aligned}$ \\
    \hline
    \end{tabular}
    \caption{The semantics of arithmetic expressions}
\end{table}


\section{Exercise 1.7}

Prove that the equations of Table 1.1 define a total function $\mathcal{A} \in \textbf{Aexp} \rightarrow
(\textbf{State} \rightarrow \textbf{Z})$: First argue that it is sufficient to prove that for each $a \in \textbf{Aexp}$
and each $s \in \textbf{State}$ there is exactly one value $v \in \textbf{Z}$ such that $\mathcal{A}\llbracket a \rrbracket s = \textbf{v}$.
Next use structural induction on the arithmetic expressions to prove that this is indeed the case.

\subsection{Demonstration by Structural Induction}

% a ::= n | x | a1 + a2 | a1 ⋆ a2 | a1 − a2

The above equations for $\mathcal{A}$ define a total function $\mathcal{A}: Aexp \rightarrow (State \rightarrow Z)$. \\

\noindent \textbf{Proof}: We have a total function $\mathcal A$, if for all arguments $a \in Aexp$

\begin{equation}
  \tag{*}
  there is exactly one number n \in \textbf{Z} such that \mathcal{A} \llbracket 
  \label{eqn:Stokes}
\end{equation}

Given an expression $a$ it can have one of 5 forms: it can be a basis element and then it is equal to the numeral $n$ or the variable $x$, or it can be a composite element and then it is equal to either the sum, the multiplication or the subtractions between two expressions $a_1$ and $a_2$. In order to prove (*), we have to consider all five possibilities.

The proof will be conducted by \textit{induction} over the \textit{structure} of the expression \textit{a}. In the base case we prove (*) for the basis elements of \textbf{Aexp},
that is for the cases where $a$ is a numeral $n$ or a variable $x$. In the \textit{induction step} we consider the composite elements of \textbf{Aexp}.
The induction hypothesis will then allow us to assume that (*) holds for the immediate constituents of $a$, which are $a_1$ and $a_2$. We shall then prove that (*) holds for $a$.
It then follows that (*) holds for all arithmetic expressions $a$ because any arithmetic expression $a$ can be constructed in that way. \\

\noindent \textbf{The case} $\mathcal{A}\llbracket n \rrbracket s = \mathcal{N} \llbracket n \rrbracket$: given that $n \in Num$ and that $\mathcal{N}$ is a total function for all arguments $n \in Num$, there is exactly one number \textbf{n} in \textbf{Z}
such that $\mathcal{A}\llbracket n \rrbracket s = \mathcal{N} \llbracket n \rrbracket$. \\

\noindent \textbf{The case} $\mathcal{A}\llbracket x \rrbracket s = s x$: given that $x \in Var$ and that the state lookup function is a total function for all arguments $x \in Var$ by definition, there is exactly one variable \textbf{x} in \textbf{Var}
such that $\mathcal{A}\llbracket x \rrbracket s = s x$. \\

\noindent \textbf{The case} $\mathcal{A}\llbracket a_1 + a_2 \rrbracket s = \mathcal{A}\llbracket a_1 \rrbracket s + \mathcal{A}\llbracket a_1 \rrbracket s$:
given that for $\mathcal{A}\llbracket a_1 \rrbracket s$ and $\mathcal{A}\llbracket a_2 \rrbracket s$ the inductive hypothesis holds, and that addition is a total function in the \textbf{Z} domain, then by structural induction, the sum of arithmetic expressions is a total function.

\noindent \textbf{The case} $\mathcal{A}\llbracket a_1 * a_2 \rrbracket s = \mathcal{A}\llbracket a_1 \rrbracket s * \mathcal{A}\llbracket a_1 \rrbracket s$:
given that for $\mathcal{A}\llbracket a_1 \rrbracket s$ and $\mathcal{A}\llbracket a_2 \rrbracket s$ the inductive hypothesis holds, and that multiplication is a total function in the \textbf{Z} domain, then by structural induction, the multiplication of arithmetic expressions is a total function.

\noindent \textbf{The case} $\mathcal{A}\llbracket a_1 - a_2 \rrbracket s = \mathcal{A}\llbracket a_1 \rrbracket s - \mathcal{A}\llbracket a_1 \rrbracket s$:
given that for $\mathcal{A}\llbracket a_1 \rrbracket s$ and $\mathcal{A}\llbracket a_2 \rrbracket s$ the inductive hypothesis holds, and that subtraction is a total function in the \textbf{Z} domain, then by structural induction, the subtraction of arithmetic expressions is a total function.


\end{document}
